\NextFile{troubleshooting.html}

\chapter{Troubleshooting}

This chapter addresses common printing issues specific to Sailfish.

\NextFile{troubleshooting-common-issues.html}

\section{Common Issues}

\subsection{The left extruder prints too far to the side}
\index{Toolhead offsets!Incorrect values}

This is a good indication that your toolhead offsets are very wrong.
Write them down and then set them each to 0~mm and see if that
corrects the problem.  You can quickly change the toolhead offsets
from the front panel using the ``Toolhead Offsets'' item under the
``Utilities'' menu.  See Section~\ref{sec:tooloff} for directions.

For assistance on calibrating your toolhead offsets, refer to
Section~\ref{sec:toolheadoffsets}.

\begin{bclogo}[logo=\bcinfo, noborder=true, couleurBarre=yellow]{Note}
If you recently installed Sailfish and are seeing this behavior, then likely
you failed to do all the setup steps from Chapter~\ref{chap:install}.  Two
of the important installation steps were to save the toolhead offsets before
installing Sailfish and then to restore them.  If you did not do those
steps, or did them incorrectly (e.g., used the wrong version of ReplicatorG),
then perhaps you should review Chapter~\ref{chap:install}.
\end{bclogo}

\subsection{Incorrect target temperatures}
\index{Temperature!Ignored}

You probably have Sailfish's ``Override Gcode Temperature'' feature
enabled. This feature ignores the temperatures requested in your gcode and
instead uses the printer's preheat temperatures. You can either set the preheat
temperatures to the temperatures you want
(Section~\ref{sec:preheatset}), or disable the temperature override
(Section~\ref{sec:override}).  For Cupcakes and Thing-o-Matics, use
ReplicatorG, MakerWare, or MakerBot Desktop to change these settings
via the Onboard Preferences windows.

\begin{bclogo}[logo=\bcinfo, noborder=true, couleurBarre=yellow]{Note}
If you recently installed Sailfish and are seeing this behavior, then likely
you failed to do all the setup steps from Chapter~\ref{chap:install}.  One
of the important installation steps is to set the printer to ``factory
defaults''.  And one of those factory defaults is to disable the ``Override
Gcode Temperature'' feature.  If it is enabled and you do not recall
enabling it, then likely you forgot that installation step.
\end{bclogo}

\subsection{Extruder cannot reach temperature}

Do you have an extruder on a dual extruder printer which cannot seem to reach
temperature during a \gls{dualstrusion} print?  You may be able to resolve
this with a change to the PID parameters for the extruder in question.

This problem is particularly common on Replicator 2X printers manufactured
during or after October 2013.  For these printers, it appears that MakerBot made
an engineering change to their Replicator 2X line and then released them without
adequate testing.  The change appears to have been made to expedite assembly
but results in too much air wash over one of the extruder's heater blocks; usually the right extruder.  With the increased air wash cooling the
extruder and the default PID settings, the extruder cannot reach temperature.

Alternatively, it may simply be that the heater core for the extruder in
question is too weak: it has too high of a resistance causing it to be of
lower wattage than normal.

In either case, you can attempt to change the PID settings for the
extruder.  Do this with ReplicatorG -- Sailfish.  Connect to the printer
over USB and then invoke the ``Onboard Preferences'' window from the
``Machine'' menu.  In the onboard preferences window, click on the
tab for the extruder in question: extruder 0 is the right extruder
and extruder 1 is the left extruder.  For the PID parameters, change the ``I''
value to 0.45 and the click the ``Commit'' button.  Once the change has
been made, power cycle the printer.  Then test the behavior.  If that does not
work, the problem may lie elsewhere.  Be very careful about experimenting
with the PID values as you do not want to destabilize the heating controls.
Also, be warned that MakerBot did not actually implement ``true'' PID
control.  Their implementation, which Sailfish follows for compatability,
is not true PID control.

\subsection{Fast, violent motion}

Should your printer seem to move too fast or violently, then likely one of
two things is occurring:

\begin{enumerate}
\item You are printing an old-style print file which has only unaccelerated
motion commands.  Such print files often use the file extension \texttt{.s3g}.
Regenerate the print file and ensure that it is a \texttt{.x3g} file.
\item The printer's acceleration is disabled.
See Section~\ref{sec:acceleration-enable} for help on checking this setting
on Replicator-style printers.  On Cupcakes and Thing-o-Matics, you will need
to use ReplicatorG, MakerWare, or MakerBot Desktop to check this setting via
machine ``Onboard Preferences''.
\end{enumerate}

Fast, violent motion is typically associated with printing at high
speeds without acceleration.  If acceleration is already enabled, make
sure you are using the correct type of build file, \texttt{.x3g}. As
of Sailfish 4.0 for the Thing-o-Matic and Cupcake and Sailfish 6.2 for
Replicators, Sailfish uses a new kind of accelerated move
command.  In ReplicatorG 0039 -- Sailfish, those new commands are used
when generating \texttt{.s3g} files, while in ReplicatorG 0040 -- Sailfish,
\texttt{.s3g} files use the old commands and \texttt{.x3g} files use the new
ones. If you use an \texttt{.s3g} file generated with any version of
ReplicatorG other than 0039 - Sailfish, it will contain the old
unaccelerated move commands and Sailfish will run them as
non-accelerated moves, resulting in violent, jerky movement that is
hard on your printer and produces poor quality prints.

If neither of these things turns out to be the issue, go to
ReplicatorG's ``Machine'' menu, select ``Onboard Preferences'', and compare
the values in the Acceleration tab to the ones shown in Table~\ref{tab:def-params}
If they look very different, you may want to reset your printer to factory
defaults.  See, for example, Section~\ref{sec:restore-settings} for directions
on accomplishing this from the front panel.  Note that this will preserve your home
and toolhead offsets.

\subsection{Prints start off the platform}

Check your printer's X and Y home offsets.  You can do this via the
printer's LCD display, if equipped, or from ReplicatorG, MakerWare, or
MakerBot Desktop.  See Section~\ref{sec:checkoffsets} for
instructions.  In that section, you will also find
Table~\ref{tab:default-offsets} which lists the default X and Y home
offsets for most types of printers.

You are looking to make sure that the X and Y offsets are not small
values or even zero or negative.  They should be close to those shown
in Table~\ref{tab:default-offsets}.  Should you change the values and
again check them, you may not see exactly the values you set.  This is a result of converting back and forth between units of
millimeters and stepper motor steps, which must be integer values; the conversion
thus introduces ``roundoff'' errors.

For Thing-o-Matics and Cupcakes, you should run the home offset calibration
procedure if you are unsure of what the correct home offset values are.
If you know what they should be, then you can set them via the ``Onboard
Preferences'' windows in ReplicatorG, MakerWare, or Desktop.

\subsection{Printer is skipping or missing steps}
\index{Print defects!Missing steps|(}
There are many possible causes for this behavior, most of them mechanical, including:

\begin{enumerate}
\item Loose pulley which slips from time to time.  Tighten set screws on all
  pulleys.
\item Faulty stepper motor wiring, or wiring not well seated in their
  connectors. Inspect wiring; reseat connectors.  Problems may only exhibit 
  at certain travel extremes.
\item Faulty endstop wiring, or wiring not well seated in their connectors.
  Inspect wiring; reseat connectors.  Problems may only exhibit at certain
  travel extremes.
\item Electrical noise picked up by endstop wiring and causing false endstop triggers.
  Route wiring away from stepper motor wiring if possible.
\item Build is too large for platform and an endstop is being triggered
  causing lost steps.
\end{enumerate}

Note that on genuine MakerBot Replicator 2 and 2X's, faulty stepper motor
and endstop wiring is not uncommon, even on brand new printers.

If you are sure that the problem is not mechanical, then it may be
that you are running an axis faster than it can tolerate.  The Z axis,
especially on a Cupcake, is particularly sensitive to this.  The
maximum feedrates allowed for each axis are specified in the machine
definition file used by ReplicatorG.  Those maximum feedrates are
specified in units of millimeters per minute.  You may need to reduce
your maximum federate.  After changing a machine definition, you will need to exit and
restart ReplicatorG so that ReplicatorG registers the change.  Then you
will need to connect your printer to ReplicatorG via USB so that
ReplicatorG can communicate the change to your printer.

However, the primary enforcement of maximum feedrates is your slicer.  Or,
more specifically, the program which translates your gcode to \gls{X3G} for
consumption by the printer's firmware.  As such, you need to also set
the maximum feedrates in your slicer.  How this is done is slicer dependent
and outside the scope of this document.
\index{Print defects!Missing steps|)}

\subsection{Leveling script does not move the extruder}

The leveling script in Sailfish (Section~\ref{sec:levelbp}) is more
versatile than that found in other firmwares: it allows you to manually
move the extruder carriage to any position you wish to check.  Moreover,
you can move it to as many positions as you want, in whatever order you
want, as many times as you want.  Just hold the extruder carriage and move
it to whereever you want to check the spacing between the extruder nozzle
and build platform.  Use the adjustment knobs under the build platform
to adjust the spacing between the build platform and nozzle.  When you are
finished, press the center button to return to the Utilities menu.

\subsection{Printer homes in the wrong direction}
\index{Homing!Wrong direction}

If you are using ReplicatorG for slicing, ensure that the ``Use default
start/end gcode'' box in the ``Generate GCode'' window is checked. If you
are using your own starting gcode, check it for errors.

If that does not solve the problem or you are not using ReplicatorG, then
inspect your starting gcode and ensure that the homing commands are
correct for your printer type as per Table~\ref{tab:homing}.  Note that
the G161 gcode command homes to the minimum endstops while a G162 command homes
to the maximum endstops.

\begin{table}[ht]
\ifpdf
\centering
\fi
\begin{tabular}{r | c c c }
\hline
& \multicolumn{3}{c}{Homing Direction}\\
Printer & X & Y & Z \\
\hline
All Replicators \& clones & max & max & min \\
Thing-o-Matics \& Cupcakes & min & min & max \\
\hline
\end{tabular}
\caption[Printer-specific homing directions]{Printer-specific homing directions}
\label{tab:homing}
\end{table}

Finally, if the homing commands appear correct in your starting gcode, then
check if the direction of motion for the axis or axes has been reversed.  If so, you can invert the direction of travel for each axis via the printer's
onboard preferences.  Inspect and change them with ReplicatorG, MakerWare,
or MakerBot Desktop.

\NextFile{troubleshooting-sd-card.html}

\section{SD Card Difficulties}

\subsection{Printer cannot read the SD card}

First, make sure that the SD card is inserted properly.  Also, check with
a good light and magnification for any debris in the SD card slot.  Vertical
SD card slots are prone to collecting fine wisps of filament and other
unwanted material which may interfere with proper operation.

If you continue to have difficulty, then try a different SD card.  It may
be that your current SD simply does not work with your printer.  Many
printers, including genuine MakerBot printers, have borderline SD card
electronics which do not perform well.  An SD card which works fine with
your computer may not work well with your printer, hence the recommendation
to try a different card.

Also, make sure that the SD card is formatted as either \gls{FAT-16} or
FAT-32. The proprietary ``eFAT'' format is not supported as it requires licensing from Microsoft.

\subsection{Missing SD card files}
\index{SD card!Missing files}

Check that the files in question end with either the \texttt{.s3g} or
\texttt{.x3g} extensions.  Sailfish recognizes files in lower, upper,
and mixed case.  Files which do not end in either of those extensions
will not be displayed, unless they are the names of folders.  Also, if the files you are looking for are in a folder,
then you will need to navigate to that folder in order to access them.
See Section~\ref{sec:sdmenu} for further details on working with SD card
folders.

Note also that filenames longer than 31 characters, filenames with
spaces, and filenames with non-US-ASCII characters may be problematic.  There
are special ``extensions'' to the FAT-16 and FAT-32 file systems for those
categories of file names. Your printer's tiny microprocessor does not have
space for the additional code to handle these extensions.

\subsection{Prints from SD cards stop without finishing}
\index{SD card!Prints fail}

If your prints from an SD card failed with the printer simply
stopping midprint without displaying an error message or clearing
the platform, then there are several likely causes, described below:

\begin{enumerate}
\item The file on the SD card is truncated.  The printer thinks the print
  is over but did not read from the print file a ``build end notification''
  command and so did not clear the build platform.  The file may have
  been truncated if the program writing it did not check for errors, as
  ReplicatorG is known to do.  When writing print files with ReplicatorG,
  it is best to write them to your local hard drive and then drag-and-drop
  the file to the SD card using your computer's file handling tools.  This
  way, if there is a problem, your computer will catch the error and
  alert you.  ReplicatorG ignores errors.

  Another cause of truncated files is a known bug in many firmwares.
  The SD card library used by many firmwares has a bug which can truncate
  SD card files.  The file will print fine at least once and then be
  truncated and print incorrectly afterwords.  This bug has been fixed in
  Sailfish 7.6 and the author of the SD card library notified.  The bug has
  not been fixed in ``stock'' firmwares.
\item The underlying gcode is truncated and incomplete.  Check the original
  gcode and ensure that it ends properly.
\item SD card read errors are occurring.  By default, SD card operations
  are not checked for errors.  You may wish to enable SD card error checking (Section~\ref{sec:sd-crc})
  and see if that resolves your difficulty.  However, in the case of a malfunctioning 
  SD card, doing so may not be sufficient.
\item An electrical anomaly caused your printer's microprocessor to reset.
  This situation is hard to detect; addressing it is outside the scope of
  this document.
\item The firmware on your printer is corrupt.  Re-install the firmware.
  Follow the directions for updating your firmware, Chapter~\ref{chap:update}.
\end{enumerate}

\NextFile{troubleshooting-dualstrusion.html}

\section{Dualstrusion}
\index{Print defects!Dualstrusion}

\Gls{dualstrusion} printing has its own set of unique issues.  A
number of the issues you might see relate to your slicer: some slicers
are much better than others at handling dualstrusion.  As of this
writing, the ``gold standard'' for dualstrusion slicing is MakerBot's
slicer found in their MakerWare and Desktop applications.  Both are
presently free and may be obtained at makerbot.com.

\subsection{Dualstrusion prints poorly aligned}

If your dualstrusion prints exhibit poor alignment between the two colors
or plastics being printed, you likely need to calibrate the nozzle spacing
(i.e., the ``toolhead offsets'').  See Section~\ref{sec:toolheadoffsets}
for detailed directions on this calibration process.

Note that the calibration process described in Section~\ref{sec:toolheadoffsets}
is only accurate to about 0.1~mm.  You are effectively picking the adjustments
with a 0.1~mm granularity.  You can, if you want, do better by printing your
own calibration prints and measuring the errors with calipers and then entering
the measured toolhead offsets manually using the ``Toolhead Offsets'' menu
as described in Section~\ref{sec:tooloff}.

\subsection{Dualstrusion prints have too much color bleed}

In general, the plastic from each extruder has a tendency to get dragged
all over the print, causing color bleed.  This is difficult to avoid.
Using ``purge walls'' helps mitigate the problem somewhat.  Try to use
a slicer which will automatically generate purge walls.  If your slicer
cannot, then consider adding them to your print before slicing.

If you feel that the firmware is contributing to the plastic oozing too
much from the idle extruder, then check to see if ``Extruder Hold'' is
enabled.  You can check it via the front panel as per
Section~\ref{sec:extruder-hold}.

\NextFile{troubleshooting-thingomatics-and-cupcakes.html}

\section{Thing-o-Matics and Cupcakes}

The following issues tend to arise on Thing-o-Matic and Cupcake printers.

\subsection{ReplicatorG shows a tool count of -1}

This is normal --- just set the extruder count to the correct count of either 1
or 2 via ReplicatorG's ``Onboard Preferences'' item found under the ``Machine''
menu.  You will first need to connect your printer to ReplicatorG over USB.

\subsection{The print starts with the extruder too far above the platform}

This typically results from running the Z axis too fast resulting in
lost steps.  As a result, the printer thinks the extruder is very close
to the build platform when in reality it is very far away.

There are two typical causes for this.  On Thing-o-Matics, it is often that
the Z-axis is binding.  Clean the Z rods, lubricate the Z axis leadscrew, and make
sure the self-aligning bearings are not binding.  They often do.  There
are numerous Thing-o-Matic modifications at thingiverse.com aimed at
addressing this common problem.  Worse comes to worst, you can reduce
the maximum feedrate for the Z axis via your slicer.

On Cupcakes, this usually occurs when printing from an SD card.  Owing to
an esoteric reason, the first motion command for a print file from an SD
card is run without acceleration.  To confirm if this is the issue, try
the same print over USB.  If printing over USB resolves the issue, then
you can either reduce the maximum Z feedrate or change your start gcode
to not ``recall home positions'' and to instead assert them with a G92
command.  For further information on why the first motion command is
not accelerated when printing from a file, see Section~\ref{sec:first-move}.

\subsection{3~mm extruder seems to be depriming too much}

Use ReplicatorG's ``Onboard Preferences'' window to enable ``Extruder
Hold''.  It is in the ``Misc'' tab. For extruders with high internal
pressure --- 3~mm extruders, typically --- it is helpful to ignore gcode
commands to temporarily disable the extruder stepper motor during a
build.  While this may not have been a problem for you previously,
Skeinforge~50 generates these commands before any \gls{travel move}.\index{Travel move}

\subsection{No plastic comes out of the extruder}

Your extruder stepper motor may be turning the wrong direction.  This
is especially true if you just switched from RPM-based gcode to Volumetric
5D gcode.  Try inverting the ``A axis'' from ReplicatorG's ``Onboard
Preferences'' window found under the ``Machine'' menu item.  You will
first need to connect your printer to ReplicatorG over USB.

\subsection{The HBP temperature climbs during a print}

There is a known bug in the Gen~4 electronics' Extruder Controller firmware.
That bug can be triggered by using an incorrectly configured
``Altshell'' plugin with Skeinforge.  If you are using Gen~4 electronics --- a
Thing-o-Matic --- and the Altshell plugin, then make sure that the
``Use M320/M321'' box is checked in the Altshell plugin's configuration.

Note that the firmware on the Extruder Controller is not Sailfish.  The
Extruder Controller has its own firmware.  The last release of that firmware
was version 3.1 back in 2011.

If you are not using Gen~4 electronics or not using the Altshell plugin,
then check the gcode in your print file.
