\NextFile{introduction.html}
\chapter{Introduction}

Sailfish began as enhanced control software for MakerBot printers,
incorporating new features intended for advanced users.  With its numerous
features, Sailfish has evolved into the firmware of choice for users of MakerBot-style printers based upon the Replicator 1 and 2 series
of 3D printers as well as the earlier Thing-o-Matic and Cupcake lines.

A 3D printer's firmware is the software which resides within the
printer and controls the printer's behavior. It is the software which
receives printing instructions from MakerWare, ReplicatorG, SD card
files, and other desktop programs and then executes them to create
your 3D print.

This documentation is intended to help you navigate the firmware on your printer, from basic setup and navigation of the \gls{LCD} screen, to advanced adjustments, updates, and the particulars of diagnosing Sailfish-specific issues.  The most recent documentation of Sailfish may be found online at:
\begin{quote}
\myhref{http://jettyfirmware.yolasite.com}{http://jettyfirmware.yolasite.com}
\end{quote}

\begin{bclogo}[logo=\bcinfo, noborder=true, couleurBarre=yellow]{Note}
Consult the documentation supplied with your printer for general printing instructions.  The \emph{Sailfish Reference Manual} only provides detailed information on the use of the Sailfish \gls{firmware} and is not intended to replace your printer's documentation.  
\end{bclogo}

\pagebreak[4]

Suggested starting points in the documentation are:
\begin{itemize}
\item Chapter~\ref{chap:basic_usage}, Basic Usage: introductory information for new 3D printer operators.
\item Chapter~\ref{chap:whatever}, Front Panel Operation: users who are familiar with 3D printers, but new to Sailfish, can begin here.
\item Chapter~\ref{chap:install}, Installing Sailfish: if you are seeking information on installing Sailfish, start here.
\end{itemize}

The Sailfish firmware is open source and builds upon earlier firmwares such as Gen~4, Grbl, and Marlin.  MakerBot's own firmware for Replicators incorporates the core of Sailfish.  Source code for Sailfish is available for inspection and download at\index{Source code}

\begin{description}
\item[] \textbf{Replicators}\
\newline
\myhref{https://github.com/jetty840/Sailfish-MightyBoardFirmware}{https://github.com/jetty840/Sailfish-MightyBoardFirmware}
\item[] \textbf{Thing-o-Matics, Cupcakes}\newline
\myhref{https://github.com/jetty840/Sailfish-G3Firmware}{https://github.com/jetty840/Sailfish-G3Firmware}
\end{description}

\noindent Important information about the compiler and tools required if you wish to build Sailfish yourself is located in the respective \texttt{doc/} directories in the \texttt{avr-gcc.markdown} file.
